\documentclass[dvipdfmx,autodetect-engine]{jsarticle}
\usepackage{tikz}

\title{サンプル文書}
\author{t180026 工藤凜}

\begin{document}
\maketitle

\begin{abstract}
数式を使用した文書を作成します。    
\end{abstract}

\section{指数と対数の関係}
指数と対数の関係いとことで言えば同じ意味のことを異なる表現で示しているということです。といっても見た目も違うので、とらえにくいと思います。まず具体的に見てみます。
\\【例】
\\※指数を使った表現
 $8=2^3$  (8は、2を3乗したもの)
\\※対数を使った表現
 log$_28=3$(2を何乗したら8になるのかというと3乗したとき)
\\この対数が便利なのは、例えば「$2^x=6$を満たす$x$は?」というときの
\\$x=log_26$というように表せるところです
\\対数の定義を確認します。
\\$a>0,a\neq1$の時、正の数$M$に対して、$a^p=M$を満たす$p$fがただ1つ定まる。この$p$の値を$a$を底とする$M$の対数といい$log_aM$と書く。
\\次に指数で表すときの利点は例えば地球と太陽の距離は約1億5000万kmであるが,これを10進法で表すと
\\150000000 [km]
となる.しかし,このように表すと一見何桁の数なのかわからず不便である.
\\このように大きな数を表すには,10を底とする指数を用いて
\\1.5×$10^8$ km
とするとよい.このように表すことによって,$10^8$の部分を見ればこの数が9桁であることをすぐに読み取れる. また,計算する上でも便利になる.指数で数を表すことについて,一般に次のことがいえる.
\\まとめると、$x\geq1$を満たす$x$は,整数部分が1桁の数$a$ (1
$ \leq a<10$)と,負でない整数$n$を使って
\\ $a\times 10^n$
\\という形ただ1通りに表すことができ,このときこの数は最高位が$n+1$桁の数である.
\\この通り指数と対数には似ているようで似ていない性質も持ち合うという関係性があります。
\end{document}